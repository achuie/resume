\documentclass[10pt, a4paper]{article}

\usepackage[left=0.75in,right=0.75in,top=0.5in,bottom=0.6in]{geometry}
\usepackage{hyperref}
\usepackage{xcolor}
\usepackage{titlesec}
\usepackage{titling}
\usepackage{bold-extra}
\usepackage{tabularx}
\usepackage{changepage}
\usepackage[parfill]{parskip}

% No page numbers.
\pagestyle{empty}

% No hyphenation.
\lefthyphenmin=62
\righthyphenmin=62

% Top level section heading.
% #1: section name
\newenvironment{aSection}[1]{
    \medskip \textbf{\uppercase{#1}}
    \smallskip
    \hrule
    \begin{list}{}{
            \setlength{\leftmargin}{1.5em}
        }
    \item[]
    }{
    \end{list}
}

% Work experience subsection.
% #1: company name
% #2: date range
% #3: job title
% #4: location
\newenvironment{expSubsection}[4]{
    \textbf{#3} \hfill {#2} \\
    {#1} \hfill \textit{#4}
    \smallskip
    \begin{list}{$\cdot$}{\leftmargin=1em}
    \itemsep -0.5em \vspace{-0.5em}
    }{
    \end{list}
    \vspace{0.5em}
}

% Project subsection.
% #1: project name
% #2: repo link
\newenvironment{projSubsection}[2]{
    {#1} \hfill {#2}
    \smallskip
    \begin{list}{$\cdot$}{\leftmargin=1em}
    \itemsep -0.5em \vspace{-0.5em}
    }{
    \end{list}
    \vspace{0.5em}
}

% Nested list with styled bullets.
\newenvironment{subList}{
    \begin{list}{\raisebox{.4ex}{\tiny$\succ$}}{\leftmargin=2em}
    \itemsep -0.5em \vspace{-0.5em}
    }{
    \end{list}
}


\renewcommand{\maketitle}{
    \begin{center}
        {\Huge\theauthor}

        \vspace{0.25em}

        \raisebox{.3ex}{\footnotesize+}81 (70)\,$\cdot$\,4387\,$\cdot$\,8863~$\diamond$
        \href{mailto:achuie@pm.me}{\textcolor{blue}{
            achuie@pm.me
        }}$\diamond$
        \href{https://www.linkedin.com/in/andrew-huie/}{\textcolor{blue}{/in/andrew-huie}}
    \end{center}

    \vspace{1em}

    \begingroup
    \addtolength\leftmargini{1.5em}
    \begin{quote}
        \textit{Software engineer with eight years of experience developing complex systems and applications. Strong
            problem solver accustomed to working in Linux environments on containerized programs. Focused on developing
            reliable, maintainable, and intuitive software.}
    \end{quote}
    \endgroup
}

\begin{document}

\title{Resume}
\author{\textsc{Andrew C. Huie}}

\maketitle

\begin{aSection}{Technical Proficiency}
    \begin{tabularx}{\textwidth}{@{}>{\bfseries}l X@{}}
        Computer Languages & Python, Rust, Go, Bash,
            C\hspace{-.05em}\raisebox{.4ex}{\tiny +}\nolinebreak\hspace{-.10em}\raisebox{.4ex}{\tiny +},
            Java, Nix, JavaScript\\
        Development Tools & Pytest, GNU/Linux (Arch \& Debian), Git, GitLab, Docker, Kubernetes, Nixpkgs
    \end{tabularx}
\end{aSection}

\begin{aSection}{Experience}
    \begin{expSubsection}
        {\textit{Mujin, Inc.} --- Autonomous industrial robotics solutions}
        {Jun 2021--Present}
        {Software Engineer}
        {Koto, Tokyo, JP}
    \item Spearheaded, designed, and implemented \textbf{Pytest} framework to validate override- \& migration-
        operations on controller configurations, reducing downtime and debugging in on-site, production deployments
    \item Lead testing efforts for customer projects, including cross-team coordination \& scheduling, simulation of
        specialized hardware, and test development
    \item Developed live controller monitoring bot in \textbf{Go}, reducing response time from days to minutes for
        thousands of deployments
        \begin{subList}
        \item Created system usage statistics module to enable automated hardware issue support
        \item Designed and implemented module to stream controller state info from \textbf{GraphQL} over websockets
        \item Automated deployment with \textbf{GitLab} and \textbf{Kubernetes}
        \end{subList}
    \item Engineered controller system simulator, enabling company-wide \textbf{test-driven development}
        \begin{subList}
        \item Simulated \textbf{QML} UI interactions to automate validation for on-site operations
        \item Emulated warehouse control systems (WCS/\textbf{PLC}) in \textbf{Python} for integration testing critical
            features
        \item Designed and programmed threaded control routines for testing complex hardware interactions
        \item Created and implemented per-project test suites of feature, corner-case, and \textbf{fault-injection}
            tests to provide guarantees for project deliverables
        \end{subList}
    \item Devised and programmed Industrial Task Language (\textbf{ITL}) control software for peripheral robotic
        hardware
    \item Developed system inspection web app using \textbf{ReactJS} \& Python for \textbf{forensic debugging}
    \item Evaluated and prototyped \textbf{Nix} for better reproducibility of builds and development, runtime, and test
        environments compared to JHBuild
    \item Set up and calibrated physical 6-axis robot test cells and successful expo demos
    \end{expSubsection}

    \begin{expSubsection}
        {\textit{Ascent Robotics, Inc.} --- Autonomous robotics technology development}
        {Sep 2017--May 2021}
        {Senior Software Engineer}
        {Shibuya, Tokyo, JP}
    \item Developed autonomous vehicle simulation suite for training/evaluating decision-making
        algorithms
        \begin{subList}
            \item \textbf{Lanelet2/OpenDrive} map generator for in-house road network format, designed to
                facilitate searching for difficult scenarios in \textbf{Rust}
            \item Emulation of perception stack output for agent training in sim environment in
                \textbf{Python}
            \item Lightweight collision sim for \textbf{MCTS} playout/rollout step in Rust
            \item High fidelity driving sim using \textbf{Unreal Engine 4} with output similar to car
                platform
        \end{subList}
    \item Conducted screening interviews for hiring candidates during growth phase of startup
    \item Created data generation pipeline for object recognition in
        \href{https://arxiv.org/abs/1805.11778}{\textcolor{blue}{\underline{publication}}}:
        \begin{adjustwidth}{1em}{}
            \vspace{-0.5em}
            Object Detection using Domain Randomization and Generative Adversarial Refinement
            of Synthetic Images \textit{ArXiv} \textbf{2018}\\
            \hspace*{1em} Fernando Camaro Nogues, \textbf{Andrew Huie}, Sakyasingha Dasgupta
        \end{adjustwidth}
    \end{expSubsection}

    \begin{expSubsection}
        {\textit{Dr. Robert Cartwright, Rice University} --- Object-oriented program development}
        {May--Sep 2016}
        {Research Assistant}
        {Houston, TX, USA}
    \item Created a new release of
        \href{http://www.drjava.org}{\textcolor{blue}{\underline{DrJava}}}, a pedagogic integrated
        development environment (IDE)
    \item Adapted the JaCoCo Java code coverage library for integrated use in DrJava
    \item Debugged JUnit integration, Find/Replace, other UI features
    \item Updated documentation with DocBook
    \end{expSubsection}

    \begin{expSubsection}
        {\textit{Dr. Dan Wallach, Rice University} --- Java TCP/IP penetration testing}
        {May--Aug 2015}
        {Research Assistant}
        {Houston, TX, USA}
    \item Inspected the security of TCP connections in Java 8, regarding the HotSpot JVM heap
    \item Ran thousands of automated trials in VMWare to stress test garbage collector
    \item Analyzed the JVM heap with VisualVM
    \item Discovered and patched security flaws
    \end{expSubsection}

    \begin{expSubsection}
        {\textit{LumaDyne Aerospace \& Scientific, LLC} --- Purpose-built scientific
        instruments}
        {Feb--Aug 2014}
        {Electrical Engineering Intern}
        {Houston, TX, USA}
    \item Designed and fabricated application-specific printed circuit boards
    \item Experience with hardware and software design tools: Multisim, Ultiboard, and LabVIEW
        \begin{subList}
            \item 3-phase brushless motor driver (PWM generator)
            \item piezoelectric crystal controller (PID control system on FPGA
                with modbus serial I/O)
            \item analog logic board
        \end{subList}
    \item Extensive soldering experience with through-hole- and surface-
        mount devices
    \end{expSubsection}

    \begin{expSubsection}
        {\textit{Salient Partners, L.P.} --- Financial assets management firm}
        {May--Aug 2013}
        {IT Intern}
        {Houston, TX, USA}
    \item Diagnosed and resolved a range of software, hardware, and network issues
    \item Deployed and repaired Dell workstations
    \end{expSubsection}
\end{aSection}

\begin{aSection}{Education} \textbf{Rice University} \hfill \textit{Houston, TX, USA}\\
    \textbf{Bachelor of Arts in Computer Science, 2016}

    \textit{Relevant Coursework:}
    \item Automata, Formal Languages, and Computability \hfill{\em Spring 2016}
    \item Principles of Programming Languages \hfill{\em Spring 2016}
    \item Computer Graphics (Game Design) \hfill{\em Spring 2016}
    \item Tools and Models in Data Science \hfill{\em Fall 2015}
    \item Operating Systems and Concurrent Programming \hfill{\em Spring 2015}
    \item Computer Security \hfill{\em Spring 2015} \item Computer Networks \hfill{\em Fall 2014}
    \item Object Oriented Programming \hfill{\em Fall 2014}
\end{aSection}

\begin{aSection}{Public Projects}
    \begin{projSubsection}
        {\textbf{scrambler}}
        {\href{https://www.github.com/achuie/scrambler}{\textcolor{blue}{\underline{github.com:achuie/scrambler}}}}
    \item[] Scramble generator for the Rubik's Cube. Random move generator as a baseline, with a more sophisticated IDA*
        solver in the works. Packaged with Nix \colorbox{lightgray}{\texttt{\$ nix run github:achuie/scrambler -- rand}}
    \end{projSubsection}

    \textbf{website}\hfill\href{https://www.github.com/achuie/website}{\textcolor{blue}{\underline{github.com:achuie/website}}}
\end{aSection}

\end{document}
